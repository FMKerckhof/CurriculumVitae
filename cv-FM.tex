\documentclass[a4paper,11pt,oneside]{article}
\usepackage{a4wide}                     % Iets meer tekst op een bladzijde
\usepackage[english]{babel}               % Voor nederlandstalige hyphenatie (woordsplitsing)
\usepackage{amsmath}                    % Uitgebreide wiskundige mogelijkheden
\usepackage{amssymb}                    % Voor speciale symbolen zoals de verzameling Z, R...
\usepackage{url}                        % Om url's te verwerken
\usepackage[round]{natbib}
\usepackage{graphicx}                   % Om figuren te kunnen verwerken
\usepackage[small,bf,hang]{caption}    % Om de captions wat te verbeteren
\usepackage{xspace}                     % Magische spaties na een commando
\usepackage[latin1]{inputenc}           % Om niet ascii karakters rechtstreeks te kunnen typen
\usepackage{float}                      % Om nieuwe float environments aan te maken. Ook optie H!
\usepackage{flafter}                    % Opdat floats niet zouden voorsteken
\usepackage{listings}                   % Voor het weergeven van letterlijke text en codelistings
\usepackage{marvosym}                   % Om het euro symbool te krijgen
\usepackage{textcomp}                   % Voor onder andere graden celsius
\usepackage{hyperref}
\usepackage{fancyhdr}                   % Voor fancy headers en footers.
\usepackage{graphics}			% Om figuren te verwerken.
%\usepackage{biblatex}
\bibliographystyle{biblioeng2}
\newcommand{\npar}{\par \vspace{2.3ex plus 0.3ex minus 0.3ex} \noindent}	% Om witruimte te krijgen tussen paragrafen
\hyphenation{stu-den-ten-ver-te-gen-woor-di-ger op-lei-dings-com-mis-sies com-pu-ter-com-mis-sies fa-cul-teits-raad 
In-ge-nieurs-We-ten-schap-pen}
\graphicspath{{figuren/}}                  % De plaars waar latex zijn figuren gaat halen.

% Nieuw commando om figuren in te voegen. Gebruik:
% \mijnfiguur[H]{width=5cm}{bestandsnaam}{Het bijschrift bij deze figuur}
\newcommand{\mijnfiguur}[4][ht]{            % Het eerste argument is standaar `ht'.
    \begin{figure}[#1]                      % Beginnen van de figure omgeving
        \begin{center}                      % Beginnen van de center omgeving
            \includegraphics[#2]{#3}        % Het eigenlijk invoegen van de figuur (2: opties, 3: bestandsnaam)
            \caption{#4\label{#3}}          % Het bijschrift (argument 4) en het label (argument 3)
        \end{center}
    \end{figure}
    }
%opening
\title{Curriculum Vitae}
\author{Frederiek - Maarten Kerckhof}

\begin{document}

\maketitle

\section{Personalia}
\begin{itemize}
  \item Naam: Kerckhof
  \item Voornaam: Frederiek - Maarten
  \item Adres: Holdaal 72A, 9000 (Gent)
  \item Bereid te verhuizen
  \item GSM: 0477/98.13.12
  \item E-mail: frederiekmaarten.kerckhof@gmail.com
  \item Geboortedatum: 14 november 1988
  \item Geboorteplaats: Brugge
\end{itemize}
\section{Onderzoeksinteresses}
Ik heb een passie voor onderzoek en ontwikkeling in data-analyse en visualisatie
methoden in diverse domeinen binnen de microbi�le ecologie. Specifiek ligt mijn 
focus op de ontwikkeling van computationeel effici�nte en kwantitative methoden,
die noodzakelijk zijn om microbi�le interacties van nabij te volgen. Ik geloof
dat een doorgedreven inzicht in microbi�le interacties essentieel is om modellen
op te bouwen die ons in staat zullen stellen om microbi�le ecosystemen beter op 
te volgen maar ook bij te sturen. Gezien microbi�le gemeenschappen essentieel 
zijn voor 
Gezien mijn gedegen wet-lab kennis op het gebied van microbi�le ecologie ben ik 
me bewust van de verschillende bronnen van "ruis", dit zorgt er voor dat ik heel
goed weet hoe een proefopzet dient te verlopen om aan het eind van de rit een 
waardevolle en statistich relevante data-interpretatie te doen.

\section{Opleiding}
\begin{itemize}
\item \textbf{2016-2017: Postdoctoraal onderzoeker in microbial resource management
 en sytnhetische microbi�le ecologie.}\\
\textbf{Funding}: Federaal Wetenschapsbeleid Belgi�, Inter-universitaire attractiepool "microbial resource managment" (BELSPO, IUAP P7/25)
    \begin{itemize}
      \item Coordineren van dagdagelijkse activiteiten van het IUAP P7/25 "micro-manager: microbial resource managment in engineered and natural ecosystems" netwerk
      \item Organisatie van interne workshops (opleidingen) voor het IUAP netwerk en CMET
      \item Begeleiding van verschillende doctoraatstudenten en 3 masterproefstudenten
      \item ARB/SILVA opleiding: "van primer naar paper" (MPI Bremen, nov 2016)
      \item TT skills opleiding (Ugent TTO): valorisatie van onderzoek, intelectuele eigendom, funding, ... (Gent, autumn 2017)
      \item EBAME3: opleiding in microbi�le computationele ecogenomics (UBO Brest, December 2017)
      \item Medelesgever Moleculair Microbi�le Technieken (1ste master bio-ingenieurswetenschappen Cel en Gen-biotechnologie + keuzestudenten, academiejaar 2017-2018)
    \end{itemize}

\item \textbf{2011-2016: Doctoraatsthesis over duurzame methanotrofie}
    \begin{itemize}
      \item Academische publicatie als eerste auteur (zie onderstaande bibliografie, \cite{kerckhof2014optimized})
      \item Co-auteur in verschillende peer-reviewed publicaties waar mijn consult als bio-informaticus en -statisticus essentieel was  (zie ook onderstaande bibliografie)
      \item Tutor van verschillende masterthesis en stagestudenten
      \item Les assistent bij de practica van Moleculair Microbi�le Technieken en Microbieel Ecologische Processen.
      \item Organiseren van interne opleidingen over statistiek, bio-informatica en versiecontrole
    \end{itemize}
\item \textbf{2011-2016: Universiteit Gent Doctoral Schools}
    \begin{itemize}
      \item Advanced academic English: conference skills
      \item IVPV specialist course: Advanced statistical methods - nonparametric methods
      \item IVPV specialist course: Advanced statistical methods - multivariate methods
      \item FLAMES specialist course: Advanced R - Programming in R and beyond
      \item Specialist course: UGent High Performance computing (Linux shell scripting, Python, HPC usage)
      \item Specialist course: Introduction to MG-RAST
    \end{itemize}
\item \textbf{2006-2011: Master Bio-ingenieurswetenschappen Cel- en Gen Biotechnologie, major computationele biologie, Universiteit gent}. \emph{Summa cum laude}
	\begin{itemize}
		\item Bachelorproef: 'Competitie en Diversiteit, een schijnbare tegenstelling'. Omtrent mathematische modellering van ecologische competitie op macro- en micro-ecologische schaal en mogelijke toepassingen (pre-emptieve kolonisatie, pre- en pro-biotica).
		\item Project Statistiek voor Genoomanalyse en Bio-Informatica 2: analyse van 454-pyrosequencing metagenomics data
		\item Masterthesis: 'Onconventionele elektron-donoren en -acceptoren voor microbi�le ecosystemen'. Fundamenteel onderzoek naar microb�le fysiologie in Bio-elektrochemische systemen (microbi�le brandstofcellen).
	\end{itemize}
	\item 2000-2006: (secundair onderwijs) Latijn - Wiskunde, Onze-Lieve-Vrouwecollege Assebroek


\end{itemize}

\section{Werkervaring}
\begin{itemize}
\item Studentenjobs
	\begin{itemize}
	\item zomers 2005-2007: Convoyeur bij Kr�fel nv. Brugge.\\ Helpen met levering en aansluiten van huishoudelijke elektronica en multimedia apparatuur.
	\item zomer 2004: Bandwerk bij Destrooper, Oostkamp. Diverse taken.
	\end{itemize}
\item Vrijwilligerswerk
	\begin{itemize}
	\item 2006-2010 Scoutsleider scouts en gidsen Vlaanderen Don Bosco, elk jaar takverantwoordelijke.
	\item 2007-2009 Materiaalmeester scouts Don Bosco.
	\item 2009-heden Groepsleider van scouts Don Bosco. Verantwoordelijk voor externe communicatie, subsidi�ring, vorming van de leidersploeg.
	\item 2010-heden Afgevaardige van mijn vereniging in de algemene Brugse jeugdraad.
	\end{itemize}
\item Extracurriculaire activiteiten
	\begin{itemize}
	\item 2004-2006 Redactie en technische staff bij OINC - TV: OLVA's informatie- nieuws- en cultuurtelevisie. Camera, geluid, montage en eindredactie.
	\item 2004-2006 Verkozen vertegenwoordiger in de leerlingenraad, werkgroepenco�rdinator.
	\item 2009-heden Klasverantwoordelijke Cel en Gen biotechnologie (examenroosters opstellen, zetelen in facultaire studentenraad).
	\item 2003-heden Geluidstechnicus op diverse kleinere shows en optredens.
	\end{itemize}
\end{itemize}

\section{Vaardigheden}
\subsection{ICT-vaardigheden}
	\begin{itemize}
	\item Algemeen: 
\begin{itemize}
	\item vertrouwd met zowel Windows en Mac Os besturingssystemen
	\item vertrouwd met MS Office: geavanceerd met Word, Powerpoint, Excel en Outlook.
	\item vertrouwd met LaTeX
\end{itemize}
 \item Programmeren: 
\begin{itemize}
	\item perl, bioperl
	\item basis Java, HTML en Visual Basic
\end{itemize}
 \item Specifieke software:
 
\begin{itemize}
	\item basiskennis bionumerics 
	\item ImageJ voor microscopische beeldanalyse
\end{itemize}
 \item Mathematische software:
 
\begin{itemize}
	\item R, bioconductor
	\item Tibco Spotfire S+
	\item TheMathWorks MATLAB
	\item Wolfram Mathematica
\end{itemize}
\end{itemize}
\subsection{Labvaardigheden}
\begin{itemize}
	\item Vertrouwd met standaard chemische analysetechnieken in microbieel onderzoek (CDW, Kjehldahl, COD, SCFA-GC, IC, ...).
	\item Vertrouwd met DGGE voor moleculaire fingerprinting van microbi�le gemeenschappen
	\item Vertrouwd met het gebruik van een flowcytometer
	\item Inleiding tot werken in BSL-3 lab
	\item Vertrouwd met standaard moleculaire analystechnieken (PCR)
\end{itemize}	

\subsection{Talenkennis}
\begin{itemize}
	\item Nederlands: moedertaal.
	\item Frans: lezen en schrijven (goed), spreken (gemiddeld)
	\item Engels: lezen, scrhijven en spreken (goed)
	\item Duits: notie
\end{itemize}

\section*{Online Presence}
\begin{itemize}
  \item \href{https://www.linkedin.com/pub/frederiek-maarten-kerckhof/26/b47/668}{LinkedIn}
  \item \href{https://www.researchgate.net/profile/Frederiek-Maarten_Kerckhof}{ResearchGate}
  \item \href{http://www.researcherid.com/ProfileView.action?SID=V2bGbhtEe1TlsfIEXBz&returnCode=ROUTER.Success&queryString=KG0UuZjN5WlUD2sX8KoC12Tw17vPT2A6ocQ5tgzRDDI\%253D\&SrcApp=CR\&Init=Yes}{ResearcherID}
  \item \href{http://orcid.org/0000-0002-4472-6810}{ORCID}
  \item Contributor to stack Exchange fora (StackOverflow, CrossValidated, Ask Ubuntu, TeX)
\end{itemize}

\renewcommand{\refname}{Academic publications and conference proceedings} %(voor article)
\renewcommand{\bibname}{Academic publications and conference proceedings} %(voor book en report)
\nocite{*}

\bibliography{citations}
%\addbibresource{citations.bib}
%\printbibliography
\end{document}
