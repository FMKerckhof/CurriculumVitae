\documentclass[a4paper,11pt,oneside]{article}
\usepackage{a4wide}                     % Iets meer tekst op een bladzijde
\usepackage[english]{babel}               % Voor nederlandstalige hyphenatie (woordsplitsing)
\usepackage{amsmath}                    % Uitgebreide wiskundige mogelijkheden
\usepackage{amssymb}                    % Voor speciale symbolen zoals de verzameling Z, R...
\usepackage{url}                        % Om url's te verwerken
\usepackage[round]{natbib}
\usepackage{graphicx}                   % Om figuren te kunnen verwerken
\usepackage[small,bf,hang]{caption}    % Om de captions wat te verbeteren
\usepackage{xspace}                     % Magische spaties na een commando
\usepackage[latin1]{inputenc}           % Om niet ascii karakters rechtstreeks te kunnen typen
\usepackage{float}                      % Om nieuwe float environments aan te maken. Ook optie H!
\usepackage{flafter}                    % Opdat floats niet zouden voorsteken
\usepackage{listings}                   % Voor het weergeven van letterlijke text en codelistings
\usepackage{marvosym}                   % Om het euro symbool te krijgen
\usepackage{textcomp}                   % Voor onder andere graden celsius
\usepackage{hyperref}
\usepackage{fancyhdr}                   % Voor fancy headers en footers.
\usepackage{graphics}			% Om figuren te verwerken.
%\usepackage{biblatex}
\bibliographystyle{biblioeng2}
\newcommand{\npar}{\par \vspace{2.3ex plus 0.3ex minus 0.3ex} \noindent}	% Om witruimte te krijgen tussen paragrafen
\hyphenation{stu-den-ten-ver-te-gen-woor-di-ger op-lei-dings-com-mis-sies com-pu-ter-com-mis-sies fa-cul-teits-raad 
In-ge-nieurs-We-ten-schap-pen}
\graphicspath{{figuren/}}                  % De plaars waar latex zijn figuren gaat halen.

% Nieuw commando om figuren in te voegen. Gebruik:
% \mijnfiguur[H]{width=5cm}{bestandsnaam}{Het bijschrift bij deze figuur}
\newcommand{\mijnfiguur}[4][ht]{            % Het eerste argument is standaar `ht'.
    \begin{figure}[#1]                      % Beginnen van de figure omgeving
        \begin{center}                      % Beginnen van de center omgeving
            \includegraphics[#2]{#3}        % Het eigenlijk invoegen van de figuur (2: opties, 3: bestandsnaam)
            \caption{#4\label{#3}}          % Het bijschrift (argument 4) en het label (argument 3)
        \end{center}
    \end{figure}
    }
%opening
\title{Curriculum Vitae}
\author{Frederiek - Maarten Kerckhof}

\begin{document}

\maketitle

\section{Personalia}
\begin{itemize}
  \item Naam: Kerckhof
  \item Voornaam: Frederiek - Maarten
  \item Adres: Holdaal 72A, 9000 (Gent)
  \item Bereid te verhuizen
  \item GSM: 0477/98.13.12
  \item E-mail: frederiekmaarten.kerckhof@gmail.com
  \item Geboortedatum: 14 november 1988
  \item Geboorteplaats: Brugge
\end{itemize}

\section{Opleiding}
\begin{itemize}
\item \textbf{2011-2016: Doctoraatsthesis over duurzame methanotrofie}
    \begin{itemize}
      \item Academische publicatie als eerste auteur (zie onderstaande bibliografie, \cite{kerckhof2014optimized})
      \item Co-auteur in verschillende peer-reviewed publicaties waar mijn consult als bio-informaticus en -statisticus essentieel was  (zie ook onderstaande bibliografie)
      \item Tutor van verschillende masterthesis en stagestudenten
      \item Les assistent bij de practica van Moleculair Microbi�le Technieken en Microbieel Ecologische Processen.
      \item Organiseren van interne opleidingen over statistiek, bio-informatica en versiecontrole
    \end{itemize}
\item \textbf{2011-2016: Universiteit Gent Doctoral Schools}
    \begin{itemize}
      \item Geavanceerd academisch Engels: conferentie skills
      \item IVPV specialisatiecursus: "Advanced statistical methods - nonparametric methods"
      \item IVPV specialisatiecursus: "Advanced statistical methods - multivariate methods"
      \item FLAMES specialisatiecursus: "Advanced R - Programming in R and beyond"
      \item Specialisatiecursus: "UGent High Performance computing (Linux shell scripting, Python, HPC usage)"
      \item Specialisatiecursus: Inleinding tot MG-RAST
    \end{itemize}
\item \textbf{2006-2011: Master Bio-ingenieurswetenschappen Cel- en Gen Biotechnologie, major computationele biologie, Universiteit gent}. \emph{Summa cum laude}
	\begin{itemize}
		\item Bachelorproef: 'Competitie en Diversiteit, een schijnbare tegenstelling'. Omtrent mathematische modellering van ecologische competitie op macro- en micro-ecologische schaal en mogelijke toepassingen (pre-emptieve kolonisatie, pre- en pro-biotica).
		\item Project Statistiek voor Genoomanalyse en Bio-Informatica 2: analyse van 454-pyrosequencing metagenomics data
		\item Masterthesis: 'Onconventionele elektron-donoren en -acceptoren voor microbi�le ecosystemen'. Fundamenteel onderzoek naar microb�le fysiologie in Bio-elektrochemische systemen (microbi�le brandstofcellen).
	\end{itemize}
	\item 2000-2006: (secundair onderwijs) Latijn - Wiskunde, Onze-Lieve-Vrouwecollege Assebroek
\end{itemize}

\section{Werkervaring}
\begin{itemize}
\item \textbf{2016-2018: Postdoctoraal onderzoeker in microbial resource management
 en synthetische microbi�le ecologie.}\\
\textbf{Funding}: Federaal Wetenschapsbeleid Belgi�, Inter-universitaire attractiepool "microbial resource managment" (BELSPO, IUAP P7/25)
    \begin{itemize}
      \item Coordineren van dagdagelijkse activiteiten van het IUAP P7/25 "micro-manager: microbial resource managment in engineered and natural ecosystems" netwerk
      \item Organisatie van interne workshops (opleidingen) voor het IUAP netwerk en CMET over softwaretools (PhenoFlow, mothur, phyloseq, R), versiecontrole (Git/GitHub) en data stewardship (submissie naar publieke databases, FAIR principes). Mede-organisator van een Software carpentry opleiding in 2017.
      \item Begeleiding van verschillende doctoraatstudenten en 3 masterproefstudenten
      \item ARB/SILVA opleiding: "van primer naar paper" (MPI Bremen, nov 2016)
      \item TT skills opleiding (Ugent TTO): valorisatie van onderzoek, intelectuele eigendom, funding, ... (Gent, autumn 2017)
      \item EBAME3: opleiding in microbi�le computationele ecogenomics (UBO Brest, December 2017)
      \item Summer school on ecological network inference and analysis: SparCC, conet, LSA, populatiedynamiek modellering, $\ldots$
      \item Medelesgever Moleculair Microbi�le Technieken (1ste master bio-ingenieurswetenschappen Cel en Gen-biotechnologie + keuzestudenten, academiejaar 2017-2018)
    \end{itemize}
\item Vrijwilligerswerk
	\begin{itemize}
	\item 2006-2010 Scoutsleider Scouts \& Gidsen Vlaanderen Don Bosco, elk jaar takverantwoordelijke.
	\item 2007-2009 Materiaalmeester scouts Don Bosco.
	\item 2009-2013 Groepsleidingsteam (met 2 anderen) van scouts Don Bosco. Verantwoordelijk voor externe communicatie, subsidi�ring, vorming van de leidersploeg van 30-40 man. Pedagogisch eindverantwoordelijke voor 200-250 leden met wekelijkse activiteiten en een jaarlijks zomerkamp.
	\item 2012-2014 Districtscommisaris in Scouts \& Gidsen Vlaanderen district 't Brugse vrije. Pedagogisch coordinator voor 12 scoutsgroepen met 350 leiders en 2000+ leden in Brugge.
	\item 2010-2014 Afgevaardige van mijn vereniging in de algemene Brugse jeugdraad.
	\end{itemize}
\item Extracurriculaire activiteiten
	\begin{itemize}
	\item 2004-2006 Redactie- en technische staff bij OINC - TV: OLVA's informatie- nieuws- en cultuurtelevisie. Camera, geluid, montage en eindredactie.
	\item 2004-2006 Verkozen vertegenwoordiger in de leerlingenraad, werkgroepenco�rdinator.
	\item 2009-2011 Klasverantwoordelijke Cel en Gen biotechnologie (examenroosters opstellen, zetelen in facultaire studentenraad).
	\end{itemize}
\end{itemize}


\section{Onderwijsvaarervaring}
\begin{itemize}
  \item 2011-2013 Practicumassistent moleculair microbi�le technieken
  \item 2013-2015 Practicumassistent microbieel ecologische processen
  \item 2016-2017 Assistent theoriecursus moleculair microbi�le technieken
  \item 2017-2018 Medelesgever moleculair microbi�le technieken
\end{itemize}


\section{Vaardigheden}
\subsection{(bio)Informatica}
Dankzij mijn major computationele biologie weet ik heel goed mijn weg te vinden
in de dataverwerking en informatica. Deze kennis heb ik verder uitgediept tijdens
mijn doctoraatsstudies en postdoc. Hieronder alvast een lijst van de software
waar ik mee vertrouwd ben.
\begin{itemize}
	\item Geavanceerde kennis van de statistische programmeertaal R en vele bioconductor en CRAN pakketten voor (bio)statistiek (Phyloseq, FlowCore, FlowViz, FlowClust, vegan, ape, ade4, $\ldots$)
	  \begin{itemize}
	    \item Mede-ontwikkelaar van het R pakket \href{https://github.com/CMET-UGent/Phenoflow_package}{PhenoFlow}
	    \item Hoofdontwikkelaar van het R pakket \href{https://github.com/CMET-UGent/MicroRaman}{MicroRaman}
	  \end{itemize}
	 \item Kennis van the Mathworks MATLAB (en simulink) voor modellering en geavanceerde wiskundige bewerkingen
	 \item Kennis van Perl/BioPerl en Python/BioPython voor scripting
	 \item Kennis van DAIME, ImageJ/FIJI en COMSTAT voor analyse van confocale beelden (bvb biofilms)
	 \item Kennis van Mothur, Qiime, dada2 en andere pipelines (zoals UPARSE) voor de analyse van microbi�le (16S) amplicon sequencing data
	 \item Kennis van Anvi'o, CONCOCT/METABAT/VizBin, IDBA-UD/MegaHIT/metaspades, MG-RAST, $\ldots$ voor data analyse van shotgun metagenomics data
	 \item Kennis van ARB, RAxML en iTOL voor geavanceerde phylogenetische inferentie (phylogenetische bomen)
	 \item Basiskennis van Bionumerics, DNAStar lasergenes, PROKKA, PathwayTools (EcoCyc/MetaCyc), CLC workbench, Wolfram Mathematica, Ruby, Java, HTML5, php en Visual basic
	 \item Opstarten en onderhoud (systeembeheer) van 3 linux servers voor algemeen gebruik binnen CMET.
	 \begin{itemize}
	    \item Installatie en updates van R pakketten voor alle gebruikers alsook Rstudio server en Shiny Server
	    \item Installatie en beheer van conda environments voor alle gebruikers
	    \item Installatie van verschillende softwarepakketten (RAxML, blast+, SparCC, MDSINE, sina, spades, $\ldots$) voor alle gebruikers
	  \end{itemize}
\end{itemize}

\subsection{Labvaardigheden}
Ik geloof dat naast mijn data-analysevaardigheden vertrouwdheid met het lab essentieel is om goed te verstaan waar de mogelijke bronnen van bias zitten en om experimental designs voor te stellen die zowel statistisch relevant als praktisch uitvoerbaar zijn.
\begin{itemize}
	\item Vertrouwd met standaard chemische analysetechnieken in microbieel onderzoek (CDW, Kjehldahl-N, COD, GC-VFA, IC, HPLC, $\ldots$).
	\item Vertrouwd met standaard moleculaire analysestechnieken (PCR, DGGE, qPCR, Illumina MiSeq amplicon data analyse)
	\item Ervaren in microbi�le flowcytometrie (op de BD FACSVerse en BD Accuri C6 instrumenten)
	\item Inleiding tot werken in BSL-3 lab
	\item Programmeren van een Tecan Freedom EVO100 pipetteerrobot
\end{itemize}

\subsection{Talenkennis}
\begin{itemize}
	\item Nederlands: moedertaal.
	\item Frans: lezen en schrijven (goed), spreken (gemiddeld)
	\item Engels: lezen, scrhijven en spreken (goed)
	\item Grieks: basiskennis
	\item Duits: notie
\end{itemize}

\section*{Online aanwezigheid}
\begin{itemize}
  \item \href{https://www.linkedin.com/pub/frederiek-maarten-kerckhof/26/b47/668}{LinkedIn}
  \item \href{https://www.researchgate.net/profile/Frederiek-Maarten_Kerckhof}{ResearchGate}
  \item \href{http://www.researcherid.com/ProfileView.action?SID=V2bGbhtEe1TlsfIEXBz&returnCode=ROUTER.Success&queryString=KG0UuZjN5WlUD2sX8KoC12Tw17vPT2A6ocQ5tgzRDDI\%253D\&SrcApp=CR\&Init=Yes}{ResearcherID}
  \item \href{http://orcid.org/0000-0002-4472-6810}{ORCID}
  \item Bijdrage aan verschillende Stack-exchange fora (StackOverflow, CrossValidated, Ask Ubuntu, TeX)
\end{itemize}

\renewcommand{\refname}{Academische publicaties en conference proceedings} %(voor article)
% \renewcommand{\bibname}{{Academische publicaties en conference proceedings} %(voor book en report)
\nocite{*}

\bibliography{citations}
%\addbibresource{citations.bib}
%\printbibliography
\end{document}
