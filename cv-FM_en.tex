\documentclass[a4paper,11pt,oneside]{article}
\usepackage{a4wide}                     % Iets meer tekst op een bladzijde
\usepackage[dutch]{babel}               % Voor nederlandstalige hyphenatie (woordsplitsing)
\usepackage{amsmath}                    % Uitgebreide wiskundige mogelijkheden
\usepackage{amssymb}                    % Voor speciale symbolen zoals de verzameling Z, R...
\usepackage{url}                        % Om url's te verwerken
\usepackage{graphicx}                   % Om figuren te kunnen verwerken
\usepackage[small,bf,hang]{caption2}    % Om de captions wat te verbeteren
\usepackage{xspace}                     % Magische spaties na een commando
\usepackage[latin1]{inputenc}           % Om niet ascii karakters rechtstreeks te kunnen typen
\usepackage{float}                      % Om nieuwe float environments aan te maken. Ook optie H!
\usepackage{flafter}                    % Opdat floats niet zouden voorsteken
\usepackage{listings}                   % Voor het weergeven van letterlijke text en codelistings
\usepackage{marvosym}                   % Om het euro symbool te krijgen
\usepackage{textcomp}                   % Voor onder andere graden celsius
\usepackage{fancyhdr}                   % Voor fancy headers en footers.
\usepackage{graphics}			% Om figuren te verwerken.
\newcommand{\npar}{\par \vspace{2.3ex plus 0.3ex minus 0.3ex} \noindent}	% Om witruimte te krijgen tussen paragrafen
\hyphenation{stu-den-ten-ver-te-gen-woor-di-ger op-lei-dings-com-mis-sies com-pu-ter-com-mis-sies fa-cul-teits-raad 
In-ge-nieurs-We-ten-schap-pen}
\graphicspath{{figuren/}}               % De plaars waar latex zijn figuren gaat halen.

% Nieuw commando om figuren in te voegen. Gebruik:
% \mijnfiguur[H]{width=5cm}{bestandsnaam}{Het bijschrift bij deze figuur}
\newcommand{\mijnfiguur}[4][ht]{            % Het eerste argument is standaar `ht'.
    \begin{figure}[#1]                      % Beginnen van de figure omgeving
        \begin{center}                      % Beginnen van de center omgeving
            \includegraphics[#2]{#3}        % Het eigenlijk invoegen van de figuur (2: opties, 3: bestandsnaam)
            \caption{#4\label{#3}}          % Het bijschrift (argument 4) en het label (argument 3)
        \end{center}
    \end{figure}
    }
%opening
\title{Curriculum Vitae}
\author{Frederiek - Maarten Kerckhof}

\begin{document}

\maketitle

\section{Personalia}
\begin{itemize}
  \item Last name: Kerckhof
  \item First name: Frederiek - Maarten
  \item Current adress: Egidius van Bredenestraat 6, 8340 Sijsele (Damme)
  \item Mobile phone: +32477/98.13.12
  \item E-mail: frederiekmaarten.kerckhof@gmail.com
  \item Date of birth: 14 November 1988
\end{itemize}
\section{Research interests}
I would very much like to conduct a research-oriented task in the environmental biotechnology domain. Due to my personal interests and training as a bioscience engineer option cellular and genetic biotechnology I consider myself qualified for a function in research and development. My social and organisatorial skills make me a good collaborator in a research team in which, if necessary, I can take a coordinating task. My main interests in environmental biotechnology lie in unravelling microbial interactions that underly the major reactions in nutrient cycling. In my belief the deeper understanding of these complex multicellular behaviour of unicellular organisms will reveal interesting perspectives in engineered ecosystems. The understanding of the operation of a microbial metabolic network  as opposed to the black-box catch-all term 'biomass' that is performing the reactions in e.g. denitrification could allow for steering of denitrification rates and outcome. I hope that I can perform state-of-the art research contributing microbial ecological knowledge on important nutrient cycles in a world where this understanding is becoming increasingly important.
\section{Education}
\begin{itemize}
\item \textbf{2006-present: Bioscience engineering in cellular and genetic biotechnology, major computational biology at Ghent University}. Masters degree expected July 2011.
	\begin{itemize}
		\item Bachelorpaper: 'Competition and diversity: apparent opposites?'. About (mathematic modelling of) ecological competition on both macro- and micro-ecological levels and possible applications (preemptive colonisation, pre- and pro-biotics).
		\item Combined project statistics for genome analysis and bio-informatics: analysis of 454-pyrosequencing metagenomics data
		\item Masters thesis: 'The impact of the physical state of electron donors and acceptors on microbial fysiology and morphology'. Fundamental research concerning microbial electron metabolism in bio-electrochemical systems (microbial fuel cells).
	\end{itemize}
\item 2000-2006: Latin - Mathematics, Onze-Lieve-Vrouwecollege Assebroek
\end{itemize}

\section{Work experience}
\begin{itemize}
\item Volunteer work
	\begin{itemize}
	\item 2006-2010: Scout leader.
	\item 2007-2009: Equiment manager in the scouts.
	\end{itemize}
\item Extracurricular activities
	\begin{itemize}
	\item 2004-2006: Editorial and technical staff at OINC-TV (secondary school). 
	\item 2004-2006: Chosen representative in the student council, work group coordinator.
	\item 2009-present: Chosen year-representative cellular and genetic biotechnology (arranging exam schedule, have a seat in the faculty's student council).
	\end{itemize}
\end{itemize}

\section{Skills}
\subsection{Informatics}
As a major in computational biology I sure know my way around computers, below are listed some of the specific software applications that I am familiar with.
\begin{itemize}
	\item R and bioconductor for statistical analysis.
	\item The Mathworks Matlab and simulink for modelling and advanced mathematics.
	\item Perl and bioperl for scripting.
	\item ImageJ and comstat for bio-imaging.
	\item Basic knowledge of Bionumerics, Wolfram Mathematica, Java, HTML and Visual Basic.
\end{itemize}
\subsection{Lab skills}
During my master's thesis at LABMET I had the opportunity to be introduced to many methods in microbiological research.
\begin{itemize}
	\item Common chemical analsysis techniques (VSS/TSS, Kjeldahl-N, COD, CDW, GC-VFA, $\ldots$).
	\item Common molecular techniques (PCR, DGGE).
	\item Use of a flowcytometer and confocal microscopy.
	\item Biosafety training for a BSL-3 laboratory.
\end{itemize}
\end{document}