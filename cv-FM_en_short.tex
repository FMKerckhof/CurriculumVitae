\documentclass[a4paper,11pt,oneside]{article}
\usepackage{a4wide}                     % Iets meer tekst op een bladzijde
\usepackage[english]{babel}               % Voor nederlandstalige hyphenatie (woordsplitsing)
\usepackage{amsmath}                    % Uitgebreide wiskundige mogelijkheden
\usepackage{amssymb}                    % Voor speciale symbolen zoals de verzameling Z, R...
\usepackage{url}                        % Om url's te verwerken
\usepackage[round]{natbib}
\usepackage{graphicx}                   % Om figuren te kunnen verwerken
\usepackage[small,bf,hang]{caption}    % Om de captions wat te verbeteren
\usepackage{xspace}                     % Magische spaties na een commando
\usepackage[latin1]{inputenc}           % Om niet ascii karakters rechtstreeks te kunnen typen
\usepackage{float}                      % Om nieuwe float environments aan te maken. Ook optie H!
\usepackage{flafter}                    % Opdat floats niet zouden voorsteken
\usepackage{listings}                   % Voor het weergeven van letterlijke text en codelistings
\usepackage{marvosym}                   % Om het euro symbool te krijgen
\usepackage{textcomp}                   % Voor onder andere graden celsius
\usepackage{hyperref}
\usepackage{fancyhdr}                   % Voor fancy headers en footers.
\usepackage{graphics}			% Om figuren te verwerken.
%\usepackage{biblatex}
\bibliographystyle{biblioeng2}
\newcommand{\npar}{\par \vspace{2.3ex plus 0.3ex minus 0.3ex} \noindent}	% Om witruimte te krijgen tussen paragrafen
\hyphenation{stu-den-ten-ver-te-gen-woor-di-ger op-lei-dings-com-mis-sies com-pu-ter-com-mis-sies fa-cul-teits-raad 
In-ge-nieurs-We-ten-schap-pen}
\graphicspath{{figuren/}}               % De plaars waar latex zijn figuren gaat halen.

% Nieuw commando om figuren in te voegen. Gebruik:
% \mijnfiguur[H]{width=5cm}{bestandsnaam}{Het bijschrift bij deze figuur}
\newcommand{\mijnfiguur}[4][ht]{            % Het eerste argument is standaar `ht'.
    \begin{figure}[#1]                      % Beginnen van de figure omgeving
        \begin{center}                      % Beginnen van de center omgeving
            \includegraphics[#2]{#3}        % Het eigenlijk invoegen van de figuur (2: opties, 3: bestandsnaam)
            \caption{#4\label{#3}}          % Het bijschrift (argument 4) en het label (argument 3)
        \end{center}
    \end{figure}
    }
%opening
\title{Curriculum Vitae}
\author{ir. Frederiek - Maarten Kerckhof}

\begin{document}

\maketitle

\section*{General information}
\subsection*{Personalia}
\rule{\textwidth}{1pt}
\begin{itemize}
  \item Last name: Kerckhof
  \item First names: Frederiek - Maarten Pieter Jozef
  \item Current adress: Brankardierstraat 44, 9000 Gent
  \item Mobile phone: +32477/98.13.12
  \item E-mail: frederiekmaarten.kerckhof@ugent.be
  \item Date and place of birth: November 14, 1988 (Bruges)
  \item Affiliation: Ghent University, Faculty of bioscience engineering, Center for Microbial ecology and technology (CMET).
  \item Adress: Coupure Links 653, 9000 Gent, Belgium
\end{itemize}

\subsection*{Education}
\rule{\textwidth}{1pt}
\begin{itemize}
\item \textbf{2011-present: Doctoral thesis on sustainable methanotrophy}\\
\textbf{Promotors}: prof. dr. ir. Nico Boon \& Dr. Kim Heylen \\
\textbf{Funding}: Ghent University GOA (BOF09/GOA/005) and Belgian Science Policy IAP (BELSPO, P7/25)
    \begin{itemize}
      \item Academic papers as first author (see bibliography below, \cite{kerckhof2014optimized} and 2 submitted)
      \item Many co-authorships as bio-informatics or statistical consultant  (see bibliography below)
      \item Guided master thesis and internship students
      \item Teaching assistant in practical exercises molecular microbial techniques and microbial ecological processes.
      \item Organizing internal courses on statistics, bio-informatics and version control
    \end{itemize}
\item \textbf{2011-present: Ghent university doctoral schools}
    \begin{itemize}
      \item IVPV specialist course: Advanced statistical methods - nonparametric methods
      \item IVPV specialist course: Advanced statistical methods - multivariate methods
      \item FLAMES specialist course: Advanced R - Programming in R and beyond
      \item Specialist course: UGent High Performance computing (Linux shell scripting, Python, HPC usage)
      \item Specialist course: Introduction to MG-RAST
    \end{itemize}
\item \textbf{2006-2011: Msc. Bioscience engineering in cellular and genetic biotechnology, major computational biology at Ghent University}. 
	\begin{itemize}
		\item Bachelorpaper: 'Competition and diversity: apparent opposites?'. About (mathematic modelling of) ecological competition on both macro- and micro-ecological levels and possible applications (preemptive colonisation, pre- and pro-biotics).
		\item Combined project statistics for genome analysis and bio-informatics: analysis of 454-pyrosequencing amplicon data
		\item Masters thesis: 'The impact of the physical state of electron donors and acceptors on microbial fysiology and morphology'. Fundamental research concerning microbial electron metabolism in bio-electrochemical systems (microbial fuel cells). - Supervisors: dr. Jan B. Arends, prof. dr. ir. Willy Verstraete \& prof. dr. ir. Nico Boon
	\end{itemize}
\item 2000-2006: Latin - Mathematics, Onze-Lieve-Vrouwecollege Assebroek
\end{itemize}



\section*{Teaching experience}
\subsection*{Practical exercises}
\rule{\textwidth}{1pt}
\begin{itemize}
  \item 2011-2013 Practical exercises molecular microbial techniques (Msc Bioscience engineering)
  \item 2013-2015 Practical exercises microbial ecological processes (Bsc Bioscience engineering)
\end{itemize}
\subsection*{Internal training}
\rule{\textwidth}{1pt}
Training organized for all collaborators within CMET to enhance the quality of research of the group.
\begin{itemize}
  \item Amplicon NGS analysis: basic training for all collaborators on use of linux command line, mothur and R.
  \item Statistics and experimental design: introduction to R, proper experimental design and one-way multiple comparisons.
  \item Version control: use of Git/GitHub for collaborative code editing.
\end{itemize}

\section*{Services}

\subsection*{Internal services}
\rule{\textwidth}{1pt}
\begin{itemize}
	\item 2011-present: server managment of CMETs core computational infrastructure: setup of 3 linux servers, user and software managment
	\item 2014-2015: organization of internal research cluster meetings on NGS sequencing and microbial ecology and interactions.
	\item 2015: Benchmarking of amplicon sequencing pipelines using mock communities - design of amplicon sequencing SOP.
	\item 2016: setup and mamangment of shiny application server, setup of Pathwaytools server
\end{itemize}
\section*{Online Presence}
\rule{\textwidth}{1pt}
\begin{itemize}
  \item \href{https://www.linkedin.com/pub/frederiek-maarten-kerckhof/26/b47/668}{LinkedIn}
  \item \href{https://www.researchgate.net/profile/Frederiek-Maarten_Kerckhof}{ResearchGate}
  \item \href{http://www.researcherid.com/ProfileView.action?SID=V2bGbhtEe1TlsfIEXBz&returnCode=ROUTER.Success&queryString=KG0UuZjN5WlUD2sX8KoC12Tw17vPT2A6ocQ5tgzRDDI\%253D\&SrcApp=CR\&Init=Yes}{ResearcherID}
  \item \href{http://orcid.org/0000-0002-4472-6810}{ORCID}
  \item Contributor to stack Exchange fora (StackOverflow, CrossValidated, Ask Ubuntu, TeX)
\end{itemize}

\newpage
\renewcommand{\refname}{Academic publications and conference proceedings} %(voor article)
\renewcommand{\bibname}{Academic publications and conference proceedings} %(voor book en report)
\nocite{*}

\bibliography{citations}
%\addbibresource{citations.bib}
%\printbibliography
\end{document}