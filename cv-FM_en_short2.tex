\documentclass[a4paper,11pt,oneside]{article}
\usepackage{a4wide}                     % Iets meer tekst op een bladzijde
\usepackage[english]{babel}               % Voor nederlandstalige hyphenatie (woordsplitsing)
\usepackage{amsmath}                    % Uitgebreide wiskundige mogelijkheden
\usepackage{amssymb}                    % Voor speciale symbolen zoals de verzameling Z, R...
\usepackage{url}                        % Om url's te verwerken
\usepackage[round]{natbib}
\usepackage{graphicx}                   % Om figuren te kunnen verwerken
\usepackage[small,bf,hang]{caption}    % Om de captions wat te verbeteren
\usepackage{xspace}                     % Magische spaties na een commando
\usepackage[latin1]{inputenc}           % Om niet ascii karakters rechtstreeks te kunnen typen
\usepackage{float}                      % Om nieuwe float environments aan te maken. Ook optie H!
\usepackage{flafter}                    % Opdat floats niet zouden voorsteken
\usepackage{listings}                   % Voor het weergeven van letterlijke text en codelistings
\usepackage{marvosym}                   % Om het euro symbool te krijgen
\usepackage{textcomp}                   % Voor onder andere graden celsius
\usepackage{hyperref}
\usepackage{fancyhdr}                   % Voor fancy headers en footers.
\usepackage{graphics}			% Om figuren te verwerken.
%\usepackage{biblatex}
\bibliographystyle{biblioeng2}
\newcommand{\npar}{\par \vspace{2.3ex plus 0.3ex minus 0.3ex} \noindent}	% Om witruimte te krijgen tussen paragrafen
\hyphenation{stu-den-ten-ver-te-gen-woor-di-ger op-lei-dings-com-mis-sies com-pu-ter-com-mis-sies fa-cul-teits-raad 
In-ge-nieurs-We-ten-schap-pen}
\graphicspath{{figuren/}}               % De plaars waar latex zijn figuren gaat halen.

% Nieuw commando om figuren in te voegen. Gebruik:
% \mijnfiguur[H]{width=5cm}{bestandsnaam}{Het bijschrift bij deze figuur}
\newcommand{\mijnfiguur}[4][ht]{            % Het eerste argument is standaar `ht'.
    \begin{figure}[#1]                      % Beginnen van de figure omgeving
        \begin{center}                      % Beginnen van de center omgeving
            \includegraphics[#2]{#3}        % Het eigenlijk invoegen van de figuur (2: opties, 3: bestandsnaam)
            \caption{#4\label{#3}}          % Het bijschrift (argument 4) en het label (argument 3)
        \end{center}
    \end{figure}
    }
%opening
\title{Curriculum Vitae}
\author{Dr. ir. Frederiek - Maarten Kerckhof}

\begin{document}

\maketitle

\section*{General information}
\subsection*{Personalia}
\rule{\textwidth}{1pt}
\begin{itemize}
  \item Last name: Kerckhof
  \item First names: Frederiek - Maarten Pieter Jozef
  \item Current adress: Holdaal 72A, 9000 Gent
  \item Mobile phone: +32477/98.13.12
  \item E-mail: frederiekmaarten.kerckhof@ugent.be
  \item Date and place of birth: November 14, 1988 (Bruges)
  \item Affiliation: Ghent University, Faculty of bioscience engineering, Center for Microbial ecology and technology (CMET).
  \item Address: Coupure Links 653, 9000 Gent, Belgium
  \item Family: Varvara Tsilia (spouse) and Phileas Kerckhof Tsilias (son, \textborn 15/09/16)
\end{itemize}

\subsection*{Experience}
\rule{\textwidth}{1pt}
\begin{itemize}
\item \textbf{2020-2021: VLAIO IM Spin-off mandate}
  \begin{itemize}
      \item VLAIO IM Spin-off mandate HBC.2019.2601: "Innovative management strategies for yeast-driven fermentations"
      \item Back-end development and support for cloud architecture
      \item see \href{https://www.kytos.be}{www.kytos.be}
  \end{itemize}

\item \textbf{2018-2020: 50\% Bio-informatician and 50\% IOF-Postdoc}
  \begin{itemize}
    \item IOF-Postdoc
    \begin{itemize}
      \item IOF StepStone project "KYTOS": on-time microbial management for sustainable bioprocesses
      \item As part of the team with a focus on applications in Fermentation industry
      \item see \href{https://www.kytos.be}{www.kytos.be}
    \end{itemize}
    \item Bio-informatician
    \begin{itemize}
      \item Organization of internal workshops for CMET
      \item Teaching molecular microbial techniques
      \item Bio-informatics support for CMET staff
      \item Data stewardship at CMET
    \end{itemize}
  \end{itemize}
\item \textbf{2016-2018: Postdoctoral research fellow on microbial resource managment and synthetic microbial ecology} \\
\textbf{Funding}: Belgian Science Policy IAP (BELSPO, P7/25), GOA Crohn (BOF17/GOA/032) 
    \begin{itemize}
      \item Coordinating day-to-day management of inter-universitary attraction poles programme P7/25 "micro-manager: microbial resource managment in engineered and natural ecosystems"
      \item Organization of internal workshops for the IUAP network and CMET
      \item Guidance of 5 PhD students and 2 masters of science students
      \item ARB/sILVA course "from primer to paper (P2P) (2016)"
      \item TT skills course (Ugent TTO): research valorization, IP, bridge funding etc.
      \item EBAME3: Workshop on Computational Microbial Ecogenomics (2017)
    \end{itemize}
\item \textbf{2011-2016: Doctoral thesis on sustainable methanotrophy}\\
\textbf{Promotors}: prof. dr. ir. Nico Boon \& Dr. Kim Heylen \\
\textbf{Funding}: Ghent University GOA (BOF09/GOA/005) and Belgian Science Policy IAP (BELSPO, P7/25)
    \begin{itemize}
      \item Academic papers as first author (see bibliography below, \cite{kerckhof2014optimized} and 2 submitted)
      \item Guided master thesis and internship students
      \item Teaching assistant in practical exercises molecular microbial techniques and microbial ecological processes.
      \item Organizing internal courses on statistics, bio-informatics and version control
    \end{itemize}
\item \textbf{2011-2016: Ghent university doctoral schools}
    \begin{itemize}
      \item IVPV specialist course: Advanced statistical methods - nonparametric methods
      \item IVPV specialist course: Advanced statistical methods - multivariate methods
      \item FLAMES specialist course: Advanced R - Programming in R and beyond
      \item Specialist course: UGent High Performance computing (Linux shell scripting, Python, HPC usage)
      \item Specialist course: Introduction to MG-RAST
    \end{itemize}
\item \textbf{2006-2011: Msc. Bioscience engineering in cellular and genetic biotechnology, major computational biology at Ghent University}. 

\end{itemize}



\section*{Teaching experience}
\subsection*{Practical exercises}
\rule{\textwidth}{1pt}
\begin{itemize}
  \item 2011-2013 Practical exercises molecular microbial techniques (Msc Bioscience engineering)
  \item 2013-2015 Practical exercises microbial ecological processes (Bsc Bioscience engineering)
\end{itemize}
\subsection*{Theory classes}
\rule{\textwidth}{1pt}
\begin{itemize}
  \item 2016-2019 Co - teacher theory course molecular microbial techniques (Msc. Bioscience engineering) 
\end{itemize}
\subsection*{Internal training}
\rule{\textwidth}{1pt}
Training organized for all collaborators within CMET to enhance the quality of research of the group.
\begin{itemize}
  \item Amplicon NGS analysis: basic training for all collaborators on use of linux command line, mothur and R.
  \item Statistics and experimental design: introduction to R, proper experimental design and one-way multiple comparisons.
  \item Version control: use of Git/GitHub for collaborative code editing.
\end{itemize}


\subsection*{Entrepreneurial skills}
\rule{\textwidth}{1pt}
In gearing up towards the perspective spin-off company Kytos I took a few courses
to make sure I would have the entrepreneurial skill-set to start a company.
\begin{itemize}
  \item Participated in business competitions: Cleantech challenge 2012 and Batle of Talents 2010
  \item UGent TechTransfer TT-skills course: essentials of IP \& funding
  \item UGent TechTransfer/Capture "disciplined entrepreneurship" course: semi-intensive program on getting from the idea to a start-up using the book "Disciplined entrepreneurship" by Bill Aulet
  \item Thriving business community BaLinCa+: 2 day financial awareness and business experience course 
  \item VOKA/UGent "from PhD to SME"
  \item Vlerick Business school "Learn to speak business"
\end{itemize}


\section*{Online Presence}
\rule{\textwidth}{1pt}
\begin{itemize}
  \item \href{https://www.linkedin.com/pub/frederiek-maarten-kerckhof/26/b47/668}{LinkedIn}
  \item \href{https://www.researchgate.net/profile/Frederiek-Maarten_Kerckhof}{ResearchGate}
  \item \href{http://www.researcherid.com/ProfileView.action?SID=V2bGbhtEe1TlsfIEXBz&returnCode=ROUTER.Success&queryString=KG0UuZjN5WlUD2sX8KoC12Tw17vPT2A6ocQ5tgzRDDI\%253D\&SrcApp=CR\&Init=Yes}{ResearcherID}
  \item \href{http://orcid.org/0000-0002-4472-6810}{ORCID}
  \item Contributor to stack Exchange fora (StackOverflow, CrossValidated, Ask Ubuntu, TeX)
\end{itemize}

\newpage
\renewcommand{\refname}{Academic publications and conference proceedings} %(voor article)
\renewcommand{\bibname}{Academic publications and conference proceedings} %(voor book en report)
\nocite{*}

\bibliography{citations}
%\addbibresource{citations.bib}
%\printbibliography
\end{document}