\documentclass[a4paper,11pt,oneside]{article}
\usepackage{a4wide}                     % Iets meer tekst op een bladzijde
\usepackage[dutch]{babel}               % Voor nederlandstalige hyphenatie (woordsplitsing)
\usepackage{amsmath}                    % Uitgebreide wiskundige mogelijkheden
\usepackage{amssymb}                    % Voor speciale symbolen zoals de verzameling Z, R...
\usepackage{url}                        % Om url's te verwerken
\usepackage{graphicx}                   % Om figuren te kunnen verwerken
\usepackage[small,bf,hang]{caption2}    % Om de captions wat te verbeteren
\usepackage{xspace}                     % Magische spaties na een commando
\usepackage[latin1]{inputenc}           % Om niet ascii karakters rechtstreeks te kunnen typen
\usepackage{float}                      % Om nieuwe float environments aan te maken. Ook optie H!
\usepackage{flafter}                    % Opdat floats niet zouden voorsteken
\usepackage{listings}                   % Voor het weergeven van letterlijke text en codelistings
\usepackage{marvosym}                   % Om het euro symbool te krijgen
\usepackage{textcomp}                   % Voor onder andere graden celsius
\usepackage{fancyhdr}                   % Voor fancy headers en footers.
\usepackage{graphics}			% Om figuren te verwerken.
\newcommand{\npar}{\par \vspace{2.3ex plus 0.3ex minus 0.3ex} \noindent}	% Om witruimte te krijgen tussen paragrafen
\hyphenation{stu-den-ten-ver-te-gen-woor-di-ger op-lei-dings-com-mis-sies com-pu-ter-com-mis-sies fa-cul-teits-raad In-ge-nieurs-We-ten-schap-pen}

%opening
\title{Curriculum Vitae}
\author{Peter Dedecker}

\begin{document}

\maketitle

\section{Personalia}
\begin{itemize}
  \item Naam: Peter Dedecker
  \item Adres: Isabellakaai 124, 9000 Gent
  \item Telefoon: 0486/15.23.20
  \item E-mail: Peter.Dedecker@VTK.UGent.be
  \item Geboortedatum: 7 november 1983
	\item Hobby's en interesses:
		\begin{itemize}
		\item Sport: badminton, fietsen
		\item Sterke interesse in wetenschap, techniek, ICT, ontwikkeling en gebruik van vrije software
		\item Ik hou er ook van de actualiteit en het reilen en zeilen in onze maatschappij op de voet te volgen.  Enige politieke interesse ontbreekt dan ook niet.  Ik ontwikkel graag mijn eigen visie op al deze zaken en heb een boontje voor mensen die, net als ik, niet graag alles zomaar over zich heen laten passeren en er een eigen kijk op nahouden.  Ik wil graag ergens een toegevoegde waarde leveren, iets verbeteren, vandaar ook mijn vele ambities en engagementen.
		\end{itemize}
\end{itemize}

\section{Opleiding}
\begin{itemize}
\item 2001-heden: Burgerlijk Ingenieur in de Computerwetenschappen optie ICT, Universiteit Gent
\item 1997-2001: Industri�le Wetenschappen, Don Bosco TI Sint-Denijs-Westrem
\item 1995-1997: Moderne Wetenschappen, Sint-Bernardus-College Oudenaarde
\end{itemize}

\section{Studentenervaringen}
\begin{itemize}
\item Stages
	\begin{itemize}
	\item zomer 2005: Stage bij Siemens Herentals: draadloze netwerken, apparatuur en (nieuwe) technologie�n, netwerktoegang op (hogesnelheids)treinen, onderzoek naar snelle handovers.
	\end{itemize}
\item Vakantiejobs
	\begin{itemize}
	\item arbeider (kalibratie van meetapparatuur) en kantoorautomatisatie (macro's) bij Alkor Draka, Oudenaarde
	\item jobstudent: arbeider (verpakking) bij Decoplast, Oudenaarde
	\end{itemize}
\item Extracurriculaire activiteiten
	\begin{itemize}
	\item 2005-2006: Ondervoorzitter Gentse StudentenRaad
	\item 2004-2005: Computerpraeses VTK Gent vzw: netwerk- en systeembeheer, website, automatisatie
	\item 2003-2004: Penningmeester VTK Gent vzw
	\item 2003: Medestichter VTK Werkgroep Vrije Software: werkgroep ter promotie van Vrije Software op de faculteit, sensibiliseren van het publiek hiervoor, ondersteunend door het geven van lessen oa Linux en de populaire LaTeX-lessen.
	\item 2002-heden: Actief (verkozen) studentenvertegenwoordiger in opleidingscommissies, computercommissies en faculteitsraad aan de Universiteit Gent - Faculteit IngenieursWetenschappen en actief in de Werkgroep Internationaal van de Vlaamse Vereniging voor Studenten, voortrekker van een aantal initiatieven in de Gentse StudentenRaad en de Facultaire Raad voor IngenieurStudenten.
	\item 2002-2005: Preventiestudent in de studentenhomes (equivalent Eerste Interventie Ploeg): EHBO, evacuaties, brandbestrijding (basis en gevorderden)
	\item 2001: Stichter burgies.tk (samenwerkingsplatform studenten eerste kan burgerlijk ingenieur, werd gebruikt voor de verspreiding van lesmateriaal, (opgeloste) oefeningen, samenvattingen,... door studenten en sommige professoren)
	\end{itemize}
\end{itemize}

\section{Vaardigheden}
\subsection{ICT-vaardigheden}
	\begin{itemize}
	\item Systeembeheer: Windows, Linux
	\item Programmeren: Java, C\#, PHP, CORBA, webservices, beperkte kennis van C(++)
	\item Algemeen: telecommunicatie, (draadloze en bekabelde) netwerken en netwerkbeheer, beveiliging (firewall, encryptie, systeembeveiliging), software ontwerp, databases, besturingssystemen, servers, web, object-georienteerd programmeren, gedistribueerde systemen
	\end{itemize}

\subsection{Kantoor}
	\begin{itemize}
	\item Uitgebreide kennis van Linux (SuSE, Mandrake, Gentoo, Ubuntu), Microsoft Windows (9X, NT, 2000, XP), OpenOffice.org, Microsoft Office, LaTeX, web, e-mail, basis van grafische bewerkingen met The Gimp, PhotoShop, PaintShopPro en Inkscape (vectorieel).
	\end{itemize}

\subsection{Talenkennis}
	Nederlands is mijn moedertaal en beheers ik dan ook uitstekend.  In het Engels kan ik zowel mondeling als schriftelijk vrij goed communiceren.  Ik kan een Franstalig gesprek of een Franstalige tekst begrijpen en zelf reageren op een middelmatig niveau.  Een woordje Duits kan ik ook nog begrijpen, maar dit is absolute basiskennis.

\subsection{Rijbewijs}
	\begin{itemize}
	\item Houder van een Europees rijbewijs B
	\end{itemize}

\section{Referenties}
\begin{itemize}
\item ing. Dirk Steel, stageleider en directe overste tijdens de stage bij Siemens NV, zomer 2005, telefoon: 0475/70.83.90
\item ir. Maarten Impens, voorzitter VTK Gent vzw 2004-2005, ondervoorzitter VTK Gent vzw 2003-2004, telefoon: 0476/75.19.53
\item ir. Stijn Baert, student-lid Raad van Bestuur Universiteit Gent 2004-2006, telefoon: 0486/49.27.52.
\end{itemize}

\end{document}
